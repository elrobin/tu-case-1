\documentclass[]{elsarticle} %review=doublespace preprint=single 5p=2 column
%%% Begin My package additions %%%%%%%%%%%%%%%%%%%
\usepackage[hyphens]{url}



\usepackage{lineno} % add
\providecommand{\tightlist}{%
  \setlength{\itemsep}{0pt}\setlength{\parskip}{0pt}}

\biboptions{sort&compress} % For natbib
\usepackage{graphicx}
\usepackage{booktabs} % book-quality tables
%%%%%%%%%%%%%%%% end my additions to header

\usepackage[T1]{fontenc}
\usepackage{lmodern}
\usepackage{amssymb,amsmath}
\usepackage{ifxetex,ifluatex}
\usepackage{fixltx2e} % provides \textsubscript
% use upquote if available, for straight quotes in verbatim environments
\IfFileExists{upquote.sty}{\usepackage{upquote}}{}
\ifnum 0\ifxetex 1\fi\ifluatex 1\fi=0 % if pdftex
  \usepackage[utf8]{inputenc}
\else % if luatex or xelatex
  \usepackage{fontspec}
  \ifxetex
    \usepackage{xltxtra,xunicode}
  \fi
  \defaultfontfeatures{Mapping=tex-text,Scale=MatchLowercase}
  \newcommand{\euro}{€}
\fi
% use microtype if available
\IfFileExists{microtype.sty}{\usepackage{microtype}}{}
\bibliographystyle{elsarticle-harv}
\usepackage{longtable}
\ifxetex
  \usepackage[setpagesize=false, % page size defined by xetex
              unicode=false, % unicode breaks when used with xetex
              xetex]{hyperref}
\else
  \usepackage[unicode=true]{hyperref}
\fi
\hypersetup{breaklinks=true,
            bookmarks=true,
            pdfauthor={},
            pdftitle={Unveiling the ecosystem of science: Towards an integrative valuation model of the many roles of scientists},
            colorlinks=false,
            urlcolor=blue,
            linkcolor=magenta,
            pdfborder={0 0 0}}
\urlstyle{same}  % don't use monospace font for urls

\setcounter{secnumdepth}{0}
% Pandoc toggle for numbering sections (defaults to be off)
\setcounter{secnumdepth}{0}
% Pandoc header



\begin{document}
\begin{frontmatter}

  \title{Unveiling the ecosystem of science: Towards an integrative valuation
model of the many roles of scientists}
    \author[a]{Nicolas Robinson-Garcia\corref{c1}}
   \ead{N.Robinson@tudelft.nl} 
   \cortext[c1]{Corresponding author}
    \author[b]{Rodrigo Costas}
  
  
    \author[b]{Thed N. van Leeuwen}
  
  
    \author[a]{Tina Nane}
  
  
      \address[a]{Applied Mathematics (DIAM), TU Delft, Delft, Netherlands}
    \address[b]{CWTS, Leiden University, Leiden, Netherlands}
  
  \begin{abstract}
  There is increasing evidence on the misuse and abuse of quantitative
  indicators in the current scientific reward system. Alternatively, more
  qualitative approaches, use of case studies or the design of indicators
  more sensitive to societal and scientific needs have been suggested. In
  this article we analyze the bases of such criticisms and motivations for
  changing the reward system by focusing on the assessment of individuals.
  We explore alternative models proposed or in use and identify common
  characteristics. Based on this we propose a valuation model by which we
  can systematically organize and prioritize performative aspects of
  scientists and consider other factors which may affect or might relevant
  for research policy. We finally test our model in a series of case
  studies based on six academic units.
  \end{abstract}
  
 \end{frontmatter}

\hypertarget{introduction}{%
\section{1. Introduction}\label{introduction}}

The ecosystem of science is defined as a system with interconnected
entities integrated in a larger social system with which a bidirectional
influence is exerted (Fortunato et al. 2018). Scientists, as drivers of
the research enterprise, are immersed in an increasingly diverse set of
tasks and activities with the purpose of producing, communicating and
transferring knowledge to society. This involves a series of skills
which go beyond their intellectual or technical capabilities (e.g.,
negotiation skills, communication skills, social engagement). The
introduction of national and supranational research evaluation systems
(Hicks 2012) carved the path to introducing measures of assessment in
the hope that they would help monitor and maximize such research
efforts. While peer review has traditionally been the main mechanism of
quality control (Bornmann 2011), there has been an increasing use of
bibliometric indicators. According to Gläser et al. (2002), this is due
to the fact that,

\begin{quote}
``{[}b{]}y applying quantitative performance indicators, actors in
science can be compared according to their performance more or less
independent of peers' judgments, and science policy can reach
conclusions seemingly independent of the scientific community to which
an actor belongs'' (pp.~29).
\end{quote}

An overemphasis on quantitative and specifically, bibliometric
indicators, has led to a situation in which academic careers are
assessed and rewarded relying solely on individuals' capacity to publish
highly cited research articles in highly ranked journals (McKiernan et
al. 2019; Wouters 2014). This unidimensional view of the research
enterprise has been harshely criticized (DORA 2014; Hicks et al. 2015;
Wilsdon et al. 2015), arguing mostly but not exclusively, on the
methodological and conceptual limitations they have when assessing
individuals. But there are more profound and serious effects derived
from current research evaluation schemes.

First, the focus on publications may lead to a task reduction, where
scientists neglect those types of work which are not recognized (Rijcke
et al. 2016). As consequence, there is a goal misplacement as the
researcher's efforts go into gaining higher scores in the measures
instead of pursuing their own research agenda. Second, quantitative
indicators are not always adequately used and introduced in research
evaluation schemes (Ràfols et al. 2016), which means that seemingly
`egalitarian' criteria can greatly hamper researchers from certain
fields or specific profiles (Gläser and Laudel 2007). Also, because such
indicators can be gamed (Fister, Fister, and Perc 2016), counteracting
rules to avoid misbehaviours can be even more harming (Robinson-Garcia
and Amat 2018). Third the focus on publishing in journals with a high
impact factor in tenure and promotion (McKiernan et al. 2019) leads to
systematic biases on the type of scientific knowledge produced. Rafols
et al. (2012) demonstrated systematic suppressing of interdisciplinary
research when using journal-based indicators for research evaluation.
Piñeiro and Hicks (2015) showed that, in the case of Spanish Sociology,
policies promoting publication in high impact journals indexed in the
Web of Science would penalize locally-oriented research. Chavarro, Tang,
and Ràfols (2017) reported that non-mainstream journals, which play an
important role in national science systems, are negatively affected by
these policies.

Finally, because a single set of criteria is defined for all scholars,
research evaluation schemes tend to promote a notion of the `excellent
scientist' who is capable of outperforming in all facets (Olmos-Peñuela,
Benneworth, and Castro-Martínez 2016), threatening the diversity of
scientific profiles researchers needed in the science ecosystem (Gläser
et al. 2002; Woolley and Robinson-Garcia 2017).

This article proposes an alternative model for evaluating scientists'
performance. This approach balances between a conceptually-informed
framework and a methodologically viable operationalization. We first
provide evidences on the specific aspects that are being negatively
affected by research evaluation schemes by reviewing previous empirical
work. After reviewing the different proposals made in the literature, we
propose a set of five performative dimensions which consider as
policy-relevant when conducting research assessment of individuals. A
key aspect of the model has to do with the networked nature of
scientific work and hence, the heterogeneity of research profiles
developed. Hence, how research trajectories develop and understanding
how specialization on specific roles might affect such categories is a
key point for our model. Also external factors affecting individuals'
performance along with effects of evaluation on personal features such
as gender, nationality or age, are considered as they can be directly or
indirectly affected by research policies and evaluation schemes. The
model is implemented in a multiple-case study of six organisational
subunits (i.e., departments, laboratories) from two Dutch universities.
This work contributes to existing research on the scientific workforce
and research careers in an inefficient evaluative context (Hammarfelt
and Rushforth 2017) which threatens diversity of scientific profiles,
and hence of outputs and impacts (Olmos-Peñuela, Benneworth, and
Castro-Martínez 2016; Woolley and Robinson-Garcia 2017).

\hypertarget{pervasive-effects-attributed-to-the-scientific-reward-system}{%
\section{2. Pervasive effects attributed to the scientific reward
system}\label{pervasive-effects-attributed-to-the-scientific-reward-system}}

Bibliometric indicators were incorporated in research evaluation
exercises in the early 1970s (De Bellis 2014). Based on the notion of
citation and productivity as predictors of recognition and academic
success (Reskin 1977; Merton 1968), their use soon expanded to from the
United States to Europe through the United Kingdom's first national
Research Assessment Exercises in the 1980s (Luukkonen 2002). Citation
and publication-based indicators were seen as an objective and
measurable way in which to assess performativity and impact. Merton's
normative framework of recognition built the grounds for the use of
citations and proxies for research quality. But it ignored the effects
the use of such indicators would have on scientists behavior and
consequently, on the production of novel societally relevant scientific
knowledge (Benedictus, Miedema, and Ferguson 2016). Furthermore,
scientific communication is drastically transforming due to the
emergence of the Open Access movement and the development of social
media and research infrastructure (i.e., data banks, repositories). This
has also affected the business model behind the publishing industry
(e.g., author fees). Hence, it is no longer feasible nor advisable to
assess qualitatively research outputs nor rely on journal or
journal-derived metrics solely (Ioannidis and Khoury 2014).

\hypertarget{effects-on-the-scientific-workforce}{%
\subsection{Effects on the scientific
workforce}\label{effects-on-the-scientific-workforce}}

The first and most obvious unintended effect is the emergence of the
so-called \emph{Publish or Perish} culture. Dalen and Henkens (2012)
indicate that scientists working in high pressure environments perceive
an external pressure to publish, which makes them pay excessive
attention only to ``tasks that benefit or glorify the individual
(publishing internationally, being cited by other scholars)'' instead of
those ``tasks that benefit larger groups'' (pp.~1291). This leads to a
task reduction (Rijcke et al. 2016) that negatively affect other tasks
expected from scholars such as teaching, knowledge transfer (Lewison
2005) or engagement with societal stakeholders (Dalen and Henkens 2012;
Robinson-Garcia, Leeuwen, and Rafols 2018).

Second, funding opportunities affect how research is conducted
(Luukkonen and Thomas 2016). Academic freedom is mediated by the
expectations the community has based on what it considers as scientific
merit (Polanyi 2000). As evaluation schemes shape such expectations,
scientists' goals can deviate from what is considered more relevant or
interesting, to what pays off best. By using journal rankings or
rewarding high citation rates, evaluation schemes pressure scientists to
modify their research agendas as well as their strategies to maximize
their efforts. For instance, an emphasis on novelty is a contributing
factor for the lack of replication studies, under-reporting of false
positives, or testing of prior published results (Nosek, Spies, and
Motyl 2012). Also, strategic publishing can lead to select topics which
are not necessarily societally relevant but that become scientific
hypes, hence becoming very profitable in terms of citations (Rousseau,
García-Zorita, and Sanz-Casado 2013).

Third, the notion of performativity becomes entangled with the
indicators used to measure it. In this case, publication output and
citation impact. A suspicion comes then from anything that deviates from
the norm, not bein able to interpret correctly those cases, as ``there
may be favoritism in some competitions and smart hiring in others''
(Abramo, D'Angelo, and Rosati 2015, 772). Furthermore, these policies
tend to emphasize and promote the \emph{superstar} or leading excellent
scientist (Olmos-Peñuela, Benneworth, and Castro-Martínez 2016),
ignoring the increasing need for team players, also referred to as the
`middle author' (Mongeon et al. 2017). As competition instead of
collaboration is promoted, those middle authors end up being excluded
from the academic career (Milojević, Radicchi, and Walsh 2018).

\hypertarget{effects-on-knowledge-production}{%
\subsection{Effects on knowledge
production}\label{effects-on-knowledge-production}}

\begin{longtable}[]{@{}lll@{}}
\toprule
\begin{minipage}[b]{0.29\columnwidth}\raggedright
Critique\strut
\end{minipage} & \begin{minipage}[b]{0.32\columnwidth}\raggedright
Evidences\strut
\end{minipage} & \begin{minipage}[b]{0.29\columnwidth}\raggedright
Reference\strut
\end{minipage}\tabularnewline
\midrule
\endhead
\begin{minipage}[t]{0.29\columnwidth}\raggedright
Misalignment between societal needs and research\strut
\end{minipage} & \begin{minipage}[t]{0.32\columnwidth}\raggedright
Desktop research and expert panel workshop\strut
\end{minipage} & \begin{minipage}[t]{0.29\columnwidth}\raggedright
Moher et al., 2018\strut
\end{minipage}\tabularnewline
\bottomrule
\end{longtable}

\begin{enumerate}
\def\labelenumi{\arabic{enumi}.}
\tightlist
\item
  Research quality: Low quality, overproduction, lack of replication,
  research integrity, fraud, hypercompetition (less collaboration),
\end{enumerate}

http://print-cfa.dk/wp-content/uploads/2019/06/ncrm-manchester-2019-jws.pdf

\begin{enumerate}
\def\labelenumi{\arabic{enumi}.}
\setcounter{enumi}{1}
\tightlist
\item
  Transparency. P-hacking, reproducibility, under-reporting of negative
  results
\end{enumerate}

Authors like Moher et al.~(2018) or Nosek (2015) refer to research
integrity, and in the case of the latter, refer to conflict of interest
of researchers as they might be interested on publication even if
results are not accurate. This would lead to Ioannidis' famous paper on
the reproducibility crisis and false results.

\begin{enumerate}
\def\labelenumi{\arabic{enumi}.}
\setcounter{enumi}{2}
\item
  Societal challenges: Not aligned with societal needs but with greater
  impact within scientific community.
\item
  Long-term agendas. Publish or perish culture is short-sighted.
\end{enumerate}

\hypertarget{towards-a-valuation-model-on-the-many-roles-of-scientists}{%
\section{3. Towards a valuation model on the many roles of
scientists}\label{towards-a-valuation-model-on-the-many-roles-of-scientists}}

\textbf{First include review on literature} Examples of dimensions can
be given from the book by Bastow, Tinkler \& Dunleavy (2014)

Bozeman, Dietz \& Gaughan (2001) present their evaluation model at the
individual level as follows:

\begin{itemize}
\tightlist
\item
  Internal resources: cognitive skills (this can be criticized as it is
  impossible to operationalize as they present it i.e., ability to
  synthesize), S\&T knowledge and context skills (learnt through
  experience). The authors indicate that each resource may have n
  dimensions.
\item
  S\&T Capital: Which has to do with the network of scientists and the
  intrinsic value they have as well as the role the individual plays in
  such network
\item
  S\&T Human Capital and Life Cycles: The authors recognize that
  researchers' profile will be different at different stages of their
  career.
\item
  Corley et al (2017) introduce a cultural dimension to the model which
  refers to variables such as gender, nationality, race, discipline or
  socio-economic status.
\end{itemize}

ACUMEN presents a portfolio, a kind of CV format designed for assessing
individual performance. It combines qualitative and quantitative
information. It offers space for a narrative where an individual tells
her own story. It then distinguishes between three aspects of an
academic's career: expertise (methods, areas of theory, etc.), outputs
(publications, patents, etc.) and impacts (citations, awards, etc.).
Furthermore, it includes an individual's age. For these three aspects it
includes quantitative indicators. The perspective here is that
individuals report their CV and the ACUMEN portfolio basically
structures and offers recommendations on how this should be provided.
Other than that it does not give clear indications on how the assessment
should be performed, it seems to rely on experts' judgment.

Look into Whitley

Look into Evaluative Inquiry

\textbf{Career trajectories} Stephan \& Levin -\textgreater{} In terms
of productivity peaks and decline

Laudel, G., Bielick, J., \& Gläser, J. (2018). `Ultimately the question
always is: ``What do I have to do to do it right?''\,' Scripts as
explanatory factors of career decisions. Human Relations,
0018726718786550. https://doi.org/10.1177/0018726718786550

\textbf{Valuation models of scientific activity}

I have changed the name of the section to valuation so that it alignes
with the ISSI poster. This means that the variables considered in the
table should also be modified and reflect on the aspects valued by each
of these models.

Interesting the model presented by Boyer (1990) and used in the paper by
Herman \& Nicholas (2019) published in El profesional de la
información. Furthermore, should consider the term \emph{facet} instead
of referring to \emph{dimensions}, it might be more accurate.

\begin{longtable}[]{@{}llll@{}}
\toprule
\begin{minipage}[b]{0.24\columnwidth}\raggedright
Evaluation model\strut
\end{minipage} & \begin{minipage}[b]{0.23\columnwidth}\raggedright
Output/outcome\strut
\end{minipage} & \begin{minipage}[b]{0.25\columnwidth}\raggedright
Unit of analysis\strut
\end{minipage} & \begin{minipage}[b]{0.16\columnwidth}\raggedright
References\strut
\end{minipage}\tabularnewline
\midrule
\endhead
\begin{minipage}[t]{0.24\columnwidth}\raggedright
Bibliometric assessment\strut
\end{minipage} & \begin{minipage}[t]{0.23\columnwidth}\raggedright
Production and impact\strut
\end{minipage} & \begin{minipage}[t]{0.25\columnwidth}\raggedright
Scalable\strut
\end{minipage} & \begin{minipage}[t]{0.16\columnwidth}\raggedright
\strut
\end{minipage}\tabularnewline
\begin{minipage}[t]{0.24\columnwidth}\raggedright
Economic analysis\strut
\end{minipage} & \begin{minipage}[t]{0.23\columnwidth}\raggedright
\strut
\end{minipage} & \begin{minipage}[t]{0.25\columnwidth}\raggedright
\strut
\end{minipage} & \begin{minipage}[t]{0.16\columnwidth}\raggedright
\strut
\end{minipage}\tabularnewline
\begin{minipage}[t]{0.24\columnwidth}\raggedright
Knowledge value alliances\strut
\end{minipage} & \begin{minipage}[t]{0.23\columnwidth}\raggedright
\strut
\end{minipage} & \begin{minipage}[t]{0.25\columnwidth}\raggedright
Communities of scholars (e.g., labs)\strut
\end{minipage} & \begin{minipage}[t]{0.16\columnwidth}\raggedright
Bozeman \& Rogers\strut
\end{minipage}\tabularnewline
\begin{minipage}[t]{0.24\columnwidth}\raggedright
Research portfolios\strut
\end{minipage} & \begin{minipage}[t]{0.23\columnwidth}\raggedright
Journal articles?\strut
\end{minipage} & \begin{minipage}[t]{0.25\columnwidth}\raggedright
\strut
\end{minipage} & \begin{minipage}[t]{0.16\columnwidth}\raggedright
Wallace \& Rafols, 2015\strut
\end{minipage}\tabularnewline
\begin{minipage}[t]{0.24\columnwidth}\raggedright
Productive interactions\strut
\end{minipage} & \begin{minipage}[t]{0.23\columnwidth}\raggedright
Process-oriented\strut
\end{minipage} & \begin{minipage}[t]{0.25\columnwidth}\raggedright
Research projects\strut
\end{minipage} & \begin{minipage}[t]{0.16\columnwidth}\raggedright
Spaape \& van Drooge, 2011\strut
\end{minipage}\tabularnewline
\begin{minipage}[t]{0.24\columnwidth}\raggedright
S\&T Human Capital\strut
\end{minipage} & \begin{minipage}[t]{0.23\columnwidth}\raggedright
\strut
\end{minipage} & \begin{minipage}[t]{0.25\columnwidth}\raggedright
Scalable\strut
\end{minipage} & \begin{minipage}[t]{0.16\columnwidth}\raggedright
Bozeman, Dietz \& Gaughan, 2001\strut
\end{minipage}\tabularnewline
\bottomrule
\end{longtable}

\hypertarget{evaluative-dimensions}{%
\subsection{Evaluative dimensions}\label{evaluative-dimensions}}

\hypertarget{scientific-engagement}{%
\subsubsection{Scientific engagement}\label{scientific-engagement}}

Collaboration ties and diversity (number of unique collaborators and
intensity), position in network, (co-authorship, acknowledgment, etc.),
strength of tie?, interdisciplinarity, application of research, member
of committees, reviewer, etc.

\hypertarget{social-engagement}{%
\subsubsection{Social engagement}\label{social-engagement}}

Non-academic collaboration ties (either through publications or not
i.e., reference patterns), some altmetric indicators (policy briefs but
also maybe twitter classes? Or potentially Ed's ABC score), spin-offs,

\hypertarget{capacity-building}{%
\subsubsection{Capacity-building}\label{capacity-building}}

Productivity, leadership, independence, scientific impact, funding

\hypertarget{trajectorybackground}{%
\subsubsection{Trajectory/Background?}\label{trajectorybackground}}

Geographic mobility, cognitive mobility, sectoral mobility (These last
three could be seen not as attributes to the researcher but their
capacity to work with diverse people), career status

\hypertarget{research-practices-cross-sectoral}{%
\subsubsection{Research practices
(Cross-sectoral)}\label{research-practices-cross-sectoral}}

OA publications, data sharing, outreach (e.g., The Conversation,
blogging, tweeting), publication of working papers, proceedings,
mentoring

Some references

Alperin, J. P., Nieves, C. M., Schimanski, L., Fischman, G. E., Niles,
M. T., \& McKiernan, E. C. (2018). How significant are the public
dimensions of faculty work in review, promotion, and tenure documents?
Recuperado de https://hcommons.org/deposits/item/hc:21015/

July 3rd, researchers, 2018\textbar{}Early career, education, H.,
science, O., evaluation, R., \& Comment, R. policy\textbar{}1. (2018,
julio 3). Making research evaluation processes in Europe more
transparent. Recuperado 20 de febrero de 2019, de
https://blogs.lse.ac.uk/impactofsocialsciences/2018/07/03/making-research-evaluation-processes-in-europe-more-transparent/

Schimanski, L. A., \& Alperin, J. P. (2018). The evaluation of
scholarship in academic promotion and tenure processes: Past, present,
and future. F1000Research, 7, 1605.
https://doi.org/10.12688/f1000research.16493.1

\hypertarget{confounding-effects}{%
\subsection{Confounding effects}\label{confounding-effects}}

Luukkonen, T., \& Thomas, D. A. (2016). The `Negotiated Space' of
University Researchers' Pursuit of a Research Agenda. Minerva, 54(1),
99-127. https://doi.org/10.1007/s11024-016-9291-z

Include here motivations of scientists -\textgreater{} puzzle, ribbon,
gold and prosocial motivations.

\hypertarget{personal-features}{%
\subsection{Personal features}\label{personal-features}}

Order dimensions from time independent to dependent. Features within
dimensions

Paper by Balen et al 2012

Also paper by Smith and Nosek 2015

\hypertarget{research-design}{%
\section{4. Research design}\label{research-design}}

\hypertarget{sample}{%
\subsection{Sample}\label{sample}}

To test the reliability of the model, we selected a technical university
and a general university. Furthermore, these subunits belong to three
different fields (Medical Sciences, Engineering Sciences and Social
Sciences), a

\hypertarget{data-sources}{%
\subsection{Data sources}\label{data-sources}}

\hypertarget{analyses}{%
\subsection{Analyses}\label{analyses}}

\hypertarget{results}{%
\section{5. Results}\label{results}}

Subsections are tentative

\hypertarget{profiling-of-scientists}{%
\subsection{Profiling of scientists}\label{profiling-of-scientists}}

\hypertarget{field-specificities}{%
\subsection{Field specificities}\label{field-specificities}}

\hypertarget{career-trajectories}{%
\subsection{Career trajectories}\label{career-trajectories}}

Table with number of scholars in each lab by source (e.g., with Google
Scholar profile) and by status.

\begin{itemize}
\item
  Scatterplots combining publications from different sources
\item
  Showcase dimensions for research teams or labs -\textgreater{}
  Internal organization and division of labor
\item
  Showcase dimensions life cycles -\textgreater{} Different stages in
  career
\item
  Attribution Plos One -\textgreater{} Scientific engagement
\item
  Differences on profiles by fields.
\item
  Agencies
\end{itemize}

\hypertarget{discussion}{%
\section{6. Discussion}\label{discussion}}

Tentative topics to discuss

\begin{itemize}
\tightlist
\item
  Diversity of roles or different career stages?
\end{itemize}

\hypertarget{references}{%
\section*{References}\label{references}}
\addcontentsline{toc}{section}{References}

\hypertarget{refs}{}
\leavevmode\hypertarget{ref-abramoDeterminantsAcademicCareer2015}{}%
Abramo, Giovanni, Ciriaco Andrea D'Angelo, and Francesco Rosati. 2015.
``The Determinants of Academic Career Advancement: Evidence from
Italy.'' \emph{Science and Public Policy} 42 (6): 761--74.
\url{https://doi.org/10.1093/scipol/scu086}.

\leavevmode\hypertarget{ref-benedictusFewerNumbersBetter2016}{}%
Benedictus, Rinze, Frank Miedema, and Mark W. J. Ferguson. 2016. ``Fewer
Numbers, Better Science.'' \emph{Nature} 538 (7626): 453--55.
\url{https://doi.org/10.1038/538453a}.

\leavevmode\hypertarget{ref-bornmannScientificPeerReview2011}{}%
Bornmann, Lutz. 2011. ``Scientific Peer Review.'' \emph{Annual Review of
Information Science and Technology} 45 (1): 197--245.
\url{http://onlinelibrary.wiley.com/doi/10.1002/aris.2011.1440450112/full}.

\leavevmode\hypertarget{ref-chavarroWhyResearchersPublish2017}{}%
Chavarro, Diego, Puay Tang, and Ismael Ràfols. 2017. ``Why Researchers
Publish in Non-Mainstream Journals: Training, Knowledge Bridging, and
Gap Filling.'' \emph{Research Policy} 46 (9): 1666--80.
\url{https://doi.org/10.1016/j.respol.2017.08.002}.

\leavevmode\hypertarget{ref-dalenIntendedUnintendedConsequences2012}{}%
Dalen, Hendrik P. van, and Kène Henkens. 2012. ``Intended and Unintended
Consequences of a Publish-or-Perish Culture: A Worldwide Survey.''
\emph{Journal of the American Society for Information Science and
Technology} 63 (7): 1282--93. \url{https://doi.org/10.1002/asi.22636}.

\leavevmode\hypertarget{ref-debellisHistoryEvolutionBiblio2014}{}%
De Bellis, Nicola. 2014. ``History and Evolution of (Biblio)Metrics.''
In \emph{Beyond Bibliometrics: Harnessing Multidimensional Indicators of
Scholarly Impact}, edited by Blaise Cronin and Cassidy R. Sugimoto,
23--44. Cambridge, MA: MIT Press.

\leavevmode\hypertarget{ref-doraSanFranciscoDeclaration2014}{}%
DORA. 2014. ``San Francisco Declaration on Research Assessment.''
\url{https://sfdora.org/}.

\leavevmode\hypertarget{ref-fisterDiscoveryCitationCartels2016}{}%
Fister, Iztok Jr, Iztok Fister, and Matjaž Perc. 2016. ``Toward the
Discovery of Citation Cartels in Citation Networks.'' \emph{Frontiers in
Physics} 4. \url{https://doi.org/10.3389/fphy.2016.00049}.

\leavevmode\hypertarget{ref-fortunatoScienceScience2018}{}%
Fortunato, Santo, Carl T. Bergstrom, Katy Börner, James A. Evans, Dirk
Helbing, Stasa Milojevic, Alexander M. Petersen, et al. 2018. ``Science
of Science.'' \emph{Science} 359 (6379): eaao0185.
\url{https://doi.org/10.1126/science.aao0185}.

\leavevmode\hypertarget{ref-glaserEvaluationEvaluators2007}{}%
Gläser, Jochen, and Grit Laudel. 2007. ``Evaluation Without
Evaluators.'' In \emph{The Changing Governance of the Sciences: The
Advent of Research Evaluation Systems}, edited by Richard Whitley and
Jochen Gläser, 127--51. Sociology of the Sciences Yearbook. Dordrecht:
Springer Netherlands. \url{https://doi.org/10.1007/978-1-4020-6746-4_6}.

\leavevmode\hypertarget{ref-glaserImpactEvaluationbasedFunding2002}{}%
Gläser, Jochen, Grit Laudel, Sybille Hinze, and Linda Butler. 2002.
``Impact of Evaluation-Based Funding on the Production of Scientific
Knowledge: What to Worry About, and How to Find Out.'' \emph{Expertise
for the German Ministry for Education and Research} 31: 2002.

\leavevmode\hypertarget{ref-hammarfeltIndicatorsJudgmentDevices2017}{}%
Hammarfelt, Björn, and Alexander D. Rushforth. 2017. ``Indicators as
Judgment Devices: An Empirical Study of Citizen Bibliometrics in
Research Evaluation.'' \emph{Research Evaluation} 26 (3): 169--80.
\url{https://doi.org/10.1093/reseval/rvx018}.

\leavevmode\hypertarget{ref-hicksPerformancebasedUniversityResearch2012}{}%
Hicks, Diana. 2012. ``Performance-Based University Research Funding
Systems.'' \emph{Research Policy} 41 (2): 251--61.
\url{http://www.sciencedirect.com/science/article/pii/S0048733311001752}.

\leavevmode\hypertarget{ref-hicksLeidenManifestoResearch2015}{}%
Hicks, Diana, Paul Wouters, Ludo Waltman, Sarah de Rijcke, and Ismael
Rafols. 2015. ``The Leiden Manifesto for Research Metrics.''
\emph{Nature} 520 (7548): 429--31.
\url{https://doi.org/10.1038/520429a}.

\leavevmode\hypertarget{ref-ioannidisAssessingValueBiomedical2014}{}%
Ioannidis, John P. A., and Muin J. Khoury. 2014. ``Assessing Value in
Biomedical Research: The PQRST of Appraisal and Reward.'' \emph{JAMA}
312 (5): 483--84. \url{https://doi.org/10.1001/jama.2014.6932}.

\leavevmode\hypertarget{ref-lewisonSCICitationsNew2005}{}%
Lewison, Grant. 2005. ``Beyond SCI Citations -- New Ways to Evaluate
Research.'' \emph{Current Science} 89 (9): 1524--30.
\url{https://www.jstor.org/stable/24110919}.

\leavevmode\hypertarget{ref-luukkonenResearchEvaluationEurope2002}{}%
Luukkonen, Terttu. 2002. ``Research Evaluation in Europe: State of the
Art.'' \emph{Research Evaluation} 11 (2): 81--84.
\url{https://doi.org/10.3152/147154402781776871}.

\leavevmode\hypertarget{ref-luukkonenNegotiatedSpaceUniversity2016}{}%
Luukkonen, Terttu, and Duncan A. Thomas. 2016. ``The `Negotiated Space'
of University Researchers' Pursuit of a Research Agenda.''
\emph{Minerva} 54 (1): 99--127.
\url{https://doi.org/10.1007/s11024-016-9291-z}.

\leavevmode\hypertarget{ref-mckiernanUseJournalImpact2019}{}%
McKiernan, Erin C., Lesley A. Schimanski, Carol Muñoz Nieves, Lisa
Matthias, Meredith T. Niles, and Juan Pablo Alperin. 2019. ``Use of the
Journal Impact Factor in Academic Review, Promotion, and Tenure
Evaluations.'' e27638v2. PeerJ Inc.
\url{https://doi.org/10.7287/peerj.preprints.27638v2}.

\leavevmode\hypertarget{ref-mertonMatthewEffectScience1968}{}%
Merton, Robert K. 1968. ``The Matthew Effect in Science the Reward and
Communication Systems of Science Are Considered.'' \emph{Science} 159
(3810): 56--63. \url{https://doi.org/10.1126/science.159.3810.56}.

\leavevmode\hypertarget{ref-milojevicChangingDemographicsScientific2018}{}%
Milojević, Staša, Filippo Radicchi, and John P. Walsh. 2018. ``Changing
Demographics of Scientific Careers: The Rise of the Temporary
Workforce.'' \emph{Proceedings of the National Academy of Sciences} 115
(50): 12616--23. \url{https://doi.org/10.1073/pnas.1800478115}.

\leavevmode\hypertarget{ref-mongeonRiseMiddleAuthor2017}{}%
Mongeon, Philippe, Elise Smith, Bruno Joyal, and Vincent Larivière.
2017. ``The Rise of the Middle Author: Investigating Collaboration and
Division of Labor in Biomedical Research Using Partial Alphabetical
Authorship.'' \emph{PLOS ONE} 12 (9): e0184601.
\url{https://doi.org/10.1371/journal.pone.0184601}.

\leavevmode\hypertarget{ref-nosekScientificUtopiaII2012}{}%
Nosek, Brian A., Jeffrey R. Spies, and Matt Motyl. 2012. ``Scientific
Utopia: II. Restructuring Incentives and Practices to Promote Truth over
Publishability.'' \emph{Perspectives on Psychological Science} 7 (6):
615--31. \url{https://doi.org/10.1177/1745691612459058}.

\leavevmode\hypertarget{ref-olmos-penuelaDoesItTake2016}{}%
Olmos-Peñuela, Julia, Paul Benneworth, and Elena Castro-Martínez. 2016.
``Does It Take Two to Tango? Factors Related to the Ease of Societal
Uptake of Scientific Knowledge.'' \emph{Science and Public Policy} 43
(6): 751--62. \url{https://doi.org/10.1093/scipol/scw016}.

\leavevmode\hypertarget{ref-pineiroReceptionSpanishSociology2015}{}%
Piñeiro, Carla López, and Diana Hicks. 2015. ``Reception of Spanish
Sociology by Domestic and Foreign Audiences Differs and Has Consequences
for Evaluation.'' \emph{Research Evaluation} 24 (1): 78--89.
\url{http://rev.oxfordjournals.org/content/24/1/78.short}.

\leavevmode\hypertarget{ref-polanyiRepublicScienceIts2000}{}%
Polanyi, Michael. 2000. ``The Republic of Science: Its Political and
Economic Theory.'' \emph{Minerva} 38 (1): 1--21.

\leavevmode\hypertarget{ref-rafolsHowJournalRankings2012}{}%
Rafols, Ismael, Loet Leydesdorff, Alice O'Hare, Paul Nightingale, and
Andy Stirling. 2012. ``How Journal Rankings Can Suppress
Interdisciplinary Research: A Comparison Between Innovation Studies and
Business \& Management.'' \emph{Research Policy}, Exploring the Emerging
Knowledge Base of 'The Knowledge Society', 41 (7): 1262--82.
\url{https://doi.org/10.1016/j.respol.2012.03.015}.

\leavevmode\hypertarget{ref-rafolsDominanceQuantitativeEvaluation2016}{}%
Ràfols, Ismael, Jordi Molas-Gallart, Diego Andrés Chavarro, and Nicolas
Robinson-Garcia. 2016. ``On the Dominance of Quantitative Evaluation in
`Peripheral' Countries: Auditing Research with Technologies of
Distance.'' SSRN Scholarly Paper ID 2818335. Rochester, NY: Social
Science Research Network.
\url{https://papers.ssrn.com/abstract=2818335}.

\leavevmode\hypertarget{ref-reskinScientificProductivityReward1977}{}%
Reskin, Barbara F. 1977. ``Scientific Productivity and the Reward
Structure of Science.'' \emph{American Sociological Review} 42 (3):
491--504. \url{https://doi.org/10.2307/2094753}.

\leavevmode\hypertarget{ref-rijckeEvaluationPracticesEffects2016}{}%
Rijcke, Sarah de, Paul F. Wouters, Alex D. Rushforth, Thomas P.
Franssen, and Björn Hammarfelt. 2016. ``Evaluation Practices and Effects
of Indicator Use---a Literature Review.'' \emph{Research Evaluation} 25
(2): 161--69. \url{https://doi.org/10.1093/reseval/rvv038}.

\leavevmode\hypertarget{ref-robinson-garciaTieneSentidoLimitar2018}{}%
Robinson-Garcia, Nicolas, and Carlos B. Amat. 2018. ``¿Tiene Sentido
Limitar La Coautoría Científica? No Existe Inflación de Autores En
Ciencias Sociales Y Educación En España.'' \emph{Revista Española de
Documentación Científica} 41 (2): 201.
\url{https://doi.org/10.3989/redc.2018.2.1499}.

\leavevmode\hypertarget{ref-robinson-garciaUsingAlmetricsContextualised2018}{}%
Robinson-Garcia, Nicolas, Thed N. van Leeuwen, and Ismael Rafols. 2018.
``Using Almetrics for Contextualised Mapping of Societal Impact: From
Hits to Networks.'' \emph{Science and Public Policy} 45 (6): 815--26.
\url{https://doi.org/10.1093/scipol/scy024}.

\leavevmode\hypertarget{ref-rousseauHbubble2013}{}%
Rousseau, Ronald, Carlos García-Zorita, and Elias Sanz-Casado. 2013.
``The H-Bubble.'' \emph{Journal of Informetrics} 7 (2): 294--300.
\url{https://doi.org/10.1016/j.joi.2012.11.012}.

\leavevmode\hypertarget{ref-wilsdonMetricTideReport2015}{}%
Wilsdon, J., Liz Allen, Eleonora Belfiore, Philip Campbell, Stephen
Curry, Steven Hill, Richard Jones, et al. 2015. ``The Metric Tide:
Report of the Independent Review of the Role of Metrics in Research
Assessment and Management.'' HEFCE.
\url{http://doi.org/10.13140/RG.2.1.4929.1363}.

\leavevmode\hypertarget{ref-woolley2014REFResults2017}{}%
Woolley, Richard, and Nicolas Robinson-Garcia. 2017. ``The 2014 REF
Results Show Only a Very Weak Relationship Between Excellence in
Research and Achieving Societal Impact.'' \emph{Impact of Social
Sciences Blog}.
\url{https://blogs.lse.ac.uk/impactofsocialsciences/2017/07/19/what-do-the-2014-ref-results-tell-us-about-the-relationship-between-excellent-research-and-societal-impact/}.

\leavevmode\hypertarget{ref-woutersCitationCultureInfrastructure2014}{}%
Wouters, Paul. 2014. ``The Citation: From Culture to Infrastructure.''
In \emph{Beyond Bibliometrics: Harnessing Multidimensional Indicators of
Scholarly Impact}, edited by Blaise Cronin and Cassidy R. Sugimoto,
47--66. Cambridge, MA: MIT Press.


\end{document}


